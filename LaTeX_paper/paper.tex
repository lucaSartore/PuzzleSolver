\documentclass{article}

% Required for inserting images
\usepackage{graphicx}

% to have more colors avelable
\usepackage{xcolor}

% to insert hiperlinks
\usepackage[colorlinks=true,linkcolor=cyan]{hyperref}

% to have better references
\usepackage[noabbrev]{cleveref}

% to use prefix for references
\usepackage{etoolbox}

% to set the margin of the page
\usepackage[a4paper, total={6.5in, 9in}]{geometry}

% to make indentations
\usepackage{changepage}

% rule for reference to fugures
\pretocmd{\thefigure}{F}{}{}

% settings for how to indend the page when neaded
\newenvironment{indented_section}
  {\adjustwidth{3em}{0pt}}
  {\endadjustwidth}

\title{Use Of Jigsaw Puzzle Solving Algorithms In The Real World}
\author{Luca Sartore}
\date{May 2023}

\begin{document}

\maketitle

\newpage

% make the table of content without the hiperlink color
{
  \hypersetup{linkcolor=black}
  \tableofcontents
}

\newpage

\section{Abstract}
The jigsaw puzzle problem has been in the eye of computer scientists for a while,
and some clever solutions have already been found. These algorithms are made to
work with a “digital” jigsaw puzzle~\ref{fig:figure_digital_puzzle},
but there aren't papers (at least not popular enough to be searchable)
that try to apply the solution
to a “real world” jigsaw puzzle~\ref{fig:figure_real_puzzle}.\newline
The problem has been tackled by some small projects. But, as said earlier,
the process and eventual challenges has never been documented by a full paper,
this wants to be the first.\newline
As a bonus the paper will also cover the creation of a user friendly app
that will be open source and free to use.

\section{Introduction}
\subsection{Classification}
This paper will focus on type 2 puzzles. A type 2 puzzle is a puzzle where the position, and the orientation of each piece is unknown. 
\subsection{Digital vs Real-World Jigsaw Puzzles}

There is  another important distinction between different types of puzzles. They can be divided into “digital” and “real world” jigsaw puzzles.


% figure of digital jigsaw puzzle    
\begin{figure}[h]
    \caption{An example of a “digital” jigsaw  puzzle}\label{fig:figure_digital_puzzle}
    \centering
    \includegraphics[height=0.25\textwidth]{pictures/digital_puzzle.png}
\end{figure}

% figure of real jigsaw puzzle    
\begin{figure}[h]
    \caption{An example of a “real world” jigsaw  puzzle}\label{fig:figure_real_puzzle}
    \includegraphics[height=0.25\textwidth]{pictures/real_puzzle.jpg}
    \centering

\end{figure}

The reason this distinction is important is because,
despite the generic concept of the puzzle not changing,
obtaining accurate matches of a piece's characteristics
is far easier with a digital puzzle,
since there are far less things that can go wrong.


% figure of a mesurement error
\begin{figure}[h]
    \caption{An example of what can go wrong when dealing with the real world}\label{fig:figure_measurement_error}
    \includegraphics[height=0.25\textwidth]{pictures/example_bad_piece.jpg}
    \centering

\end{figure}

\section{Previous Literature}
This section will analyze 3 different algorithms that have been proposed
as a solution of type 2 puzzles. The objective is to understand the
strengths and the weaknesses of each one, to build up some knowledge
that will be useful for the next section.

\subsection{General Structure Of The Algorithms}

All the algorithms that will be analyzed are composed of 3 sub algorithms:


\begin{indented_section}

    \subsubsection{The Splitter:} This component takes as input one or more images
    containing all the pieces. It then split all the pieces from each other,
    9and split each piece into his four sides.\label{document:splitter}

    \subsubsection{The Comparator:} This component compares each side with all the
    others, in order to understand whether they match or not.\newline
    There are two distinct kinds of “Comparator” algorithms:
    The “Binary Comparison” and the “Non Binary Comparison”.
    As the name suggests when comparing two sides with a “Binary Comparison”
    the result can either be 0 (they do not match) or 1 (they match).
    In contrast a “Non Binary Comparison” can give any value between 0 and 1.
    This allows states of uncertainty to be represented.\label{document:comparator}

    \subsubsection{The Solver:} This component  uses the information provided by the
    Comparator~\ref{document:comparator}, and tries to find a solution
    (i.e.\ a position and an orientation for each piece)
    that is the most likely to be correct.\label{document:solver}

\end{indented_section}

\subsection{Solving Jigsaw Puzzles By The Graph Connection Laplacian~\cite{GCL}}
This algorithm falls in the “binary comparison” category~\ref{document:comparator}.
And can be used to understand the strength and weaknesses
of this approach.\newline
The main benefit of this approach is speed,  having a binary comparison allows
some specific optimization, in particular the use of some graph theory techniques.\newline
The  algorithm has a time complexity of ruffly \(O(N^2)\). Which is in line with the theoretical
minimum for jigsaw puzzle solving, according to the study:
“No easy puzzles: Hardness results for jigsaw puzzles~\cite{ON2Claim}”.\newline
The negative aspect might be the accuracy, since by using only two states
(match or not match) some informations are lost, compared to a non binary comparison.
To quantify accuracy the paper used the “neighbors comparison” metric,
which “calculates the percentage of pairs of image patches that are matched correctly”.\newline
Is important to note that the discussed algorithm works with 
digital puzzles~\ref{fig:figure_digital_puzzle}, and this gives it the advantage of having more precise data.\newline
Whether this algorithm could work with a real world puzzle~\ref{fig:figure_real_puzzle} will be discussed in section DOTO INSERT RFERENCE.



\subsection{A Genetic Algorithm-Based Solver for Very Large Jigsaw Puzzles~\cite{GA}}


This paper applies a genetic algorithm to the jigsaw puzzle problem, with some promising results.\newline
The algorithm seems to have a \(O(N^2)\) time complexity, with time slightly lower than the previous example
(But it is impossible to know for sure which one is faster, given that they didn't specify the hardware used).\newline
The paper has used the same method to evaluate accuracy that the previous one used
(neighbors comparison) so it is easy to compare them.\newline
The algorithm falls in the “Non binary comparison”~\ref{document:comparator};
this should give it an advantage, since it has more data to work with.\newline
Unfortunately this advantage does not compensate for the worst precision
of the algorithm itself, and the accuracy results are equivalent,
if not slightly worse than the previous algorithm.\newline
It is important to keep in mind that for the very nature of genetic algorithms,
it might be possible that the accuracy would have been better if they had allowed
it to run for more generations.\newline
Is also important to note that the discussed algorithm works with digital puzzles,
and this gives it the advantage of having more precise data.\newline
Whether this algorithm could work with a real world puzzle will be discussed
in section DOTO INSERT RFERENCE.




% referneces page
\newpage
\begin{thebibliography}{9}

  %Graph Connection Laplacian
  \bibitem{GCL}
    Vahan Huroyan, Gilad   Lerman and Hau-Tieng Wu,
    Solving Jigsaw Puzzles By The Graph Connection Laplacian,
    2020.
    \url{https://arxiv.org/pdf/1811.03188.pdf}.
  
  % Genetic algorighm
  \bibitem{GA}
    Dror Sholomon, Omid David and Nathan S. Netanyahu,
    A Genetic Algorithm-Based Solver for Very Large Jigsaw Puzzles,
    2013.
    \url{https://openaccess.thecvf.com/content_cvpr_2013/papers/Sholomon_A_Genetic_Algorithm-Based_2013_CVPR_paper.pdf}.
  
    % Claim of best solution O(N^2)
    \bibitem{ON2Claim}
    Michael Brand,
    No easy puzzles: Hardness results for jigsaw puzzles,
    2015.
    \url{https://www.sciencedirect.com/science/article/pii/S0304397515001607}.
  
\end{thebibliography}

\end{document}
